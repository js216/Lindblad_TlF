\documentclass{article}
\usepackage[cp1250]{inputenc}
\usepackage{graphicx}
\usepackage{amsmath,amsthm,amssymb,mathrsfs}
\usepackage{mathtools}
\usepackage{physics}
\usepackage{subfig}
\usepackage[titletoc]{appendix}
\usepackage{array}
\usepackage{multirow}
\usepackage{arydshln}
\usepackage{lscape}
\usepackage{tikz} %Package for making energy level diagrams
\usetikzlibrary{calc,arrows,decorations.pathmorphing,intersections}
\usepackage{slashbox}
\usepackage{etoolbox}
\usepackage[left = 2cm, right=2cm, top=2.5cm, bottom=2.5cm, headsep=1.2cm]{geometry}
\usepackage{gensymb}
\usepackage[backend=bibtex,style=numeric,citestyle=nature, natbib=true]{biblatex} % Use the bibtex backend with the authoryear citation style (which resembles APA)
\usepackage{amsmath}
\usepackage{outlines} %Package for facilitating use of nested bullet points
\usepackage{multirow} %Package for tables to make cells that span multiple rows
\usepackage{makecell} %Package for linebreaks within table cells
\usepackage{float}
\usepackage{siunitx} %Dealing with inline math breaking over multiple lines
\usepackage{setspace} %Package for seting linspacing

\setcounter{MaxMatrixCols}{20}
\usetikzlibrary{positioning}
\usetikzlibrary{matrix}
\usetikzlibrary{arrows,decorations.pathmorphing}
\numberwithin{equation}{section}
\numberwithin{table}{section}
\DeclarePairedDelimiter{\ceil}{\lceil}{\rceil}

\patchcmd{\bordermatrix}{\begingroup}{\begingroup\openup2\jot}{}{}
\makeatletter
\patchcmd{\bordermatrix}
  {\vcenter{\kern-\ht\@ne\unvbox\z@\kern-\baselineskip}}
  {\vcenter{\kern-\ht\@ne\unvbox\z@\kern-\baselineskip\kern3pt}}
  {}{}
\makeatother

\newcommand{\tj}[6]{ \begin{pmatrix}
   #1 & #2 & #3 \\
   #4 & #5 & #6 
  \end{pmatrix}}
 
\newcommand{\sj}[6]{ \begin{Bmatrix}
   #1 & #2 & #3 \\
   #4 & #5 & #6 
  \end{Bmatrix}}

%Info for titlepage
\title{C'(Tl) -term in uncoupled basis}
\author{Oskari Timgren}

\begin{document}
	\maketitle
	\doublespacing

This document goes through the derivation of the matrix elements of the C'(Tl)-term as defined by Brown et al. in "A determination of fundamental Zeeman parameters for the OH radical" (1978):
\begin{align}
	H_{nsr}' = C_I' \sum_{q = \pm1} \exp(-2iq\phi) \frac{1}{2} \left[T^2_{2q}(I, J-S) + T^2_{2q}(J-S,I)\right] 
\end{align} 

We'll be using a basis set where the nuclear spin I is decoupled from J: $ \ket{\eta; J,\Omega, m_J, I_1, m_1, I_2, m_2} $. Since we're only interested in matrix elements for the Tl-spin, the quantum numbers for the fluorine will be suppressed; the selection rules for them will be $ \delta_{I_2, I_2'} \delta_{m_2, m_2'} $. We will also drop the electron spin operator S since it only couples to different electronic states, and thus does not contribute significantly. We can thus write the operators as $ T^2_{\pm2}(I, J) = T^1_{\pm1}(I)T^1_{\pm1}(J) $. To evaluate the matrix elements we'll use the following relations:

\begin{outline}
	\1 B\&C 5.162:
	\begin{align}
		&\bra{J',\Omega', m_J', I_1', m_1'}T^1_q(J)\ket{J,\Omega, m_J, I_1, m_1} \\
		& = (-1)^{J'-\Omega'-q} \tj{J'}{1}{J'}{-\Omega'}{-q}{\Omega} \left[J'\left(J'+1\right)\left(2J'+1\right)\right]^{1/2} \delta_{J,J'} \delta_{m_J,m_J'} \delta_{I_1, I_1'} \delta_{m_1, m_1'}
	\end{align}
	\1 B\&C5.144 (transforming operator to lab frame),  B\&C5.172 (W-E theorem), 5.179 (reduced ME for 1st rank tensor)
	\begin{align}
		&\bra{J',\Omega', m_J', I_1', m_1'}T^1_q(I)\ket{J,\Omega, m_J, I_1, m_1} \\
		=& \sum_p (-1)^{p-q}\bra{J',\Omega', m_J', I_1', m_1'}\mathit{D}^{(1)*}_{-p,-q}T^1_p(I)\ket{J,\Omega, m_J, I_1, m_1} \\
		=& \sum_p (-1)^{p-q} (-1)^{J'-\Omega'} \left[\left(2J'+1\right)\left(2J+1\right)\right]^{1/2} \tj{J'}{1}{J}{-m_J'}{-p}{m_J} \tj{J'}{1}{J}{-\Omega'}{-q}{\Omega} \\
		&\qquad\qquad(-1)^{I_1'-m_1'} \tj{I'}{1}{I}{-m'_1}{p}{m_1}\left[I_1\left(I_1+1\right)\left(2I_1+1\right)  \right]\delta_{I_1, I_1'} \\
		=&  (-1)^{p-q+J'-\Omega'+I_1'-m_1'} \left[\left(2J'+1\right)\left(2J+1\right)I_1\left(I_1+1\right)\left(2I_1+1\right)\right]^{1/2} \\
		&\qquad \tj{J'}{1}{J}{-m_J'}{-p}{m_J} \tj{J'}{1}{J}{-\Omega'}{-q}{\Omega} \tj{I'}{1}{I}{-m'_1}{p}{m_1}
	\end{align}
\end{outline}

\end{document}